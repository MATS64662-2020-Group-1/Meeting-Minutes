\documentclass{article}
\usepackage{amsmath}
\usepackage{graphicx}
\usepackage[margin=2cm, bindingoffset=1cm, inner=2cm]{geometry}
\usepackage{hyperref}
\usepackage{textcomp}
\usepackage{float}

\title{\bf Meeting Minutes 07/04/2020}
\author{Group 1}

\begin{document}
\maketitle


\section{RSEP Tutorial 2}


- Titanium images are on Git, license grade, no attachment images \newline\newline
- Two microstructures: lamellar and globular microstructures\newline\newline
- Can work with composites or porous materials\newline\newline
- Report details added to Git\newline\newline
- Mostly graded on GitHub activity (65\%) e.g. raising issues, commits, discussions etc.\newline\newline
- Report summarising everything we’ve done\newline\newline
- Report split into group and individual parts\newline\newline
- Group report: going over software, problem solved, technical report\newline\newline
- Individual report: reflective, what was our specific contribution, talk about what we did right/wrong in project, successful? Be critical\newline\newline
- Free format on the report\newline\newline
- Document meetings, minutes etc.\newline\newline
- Report due on 19th June\newline\newline
- Get sfepy examples running -> already done\newline\newline
- Gmsh, Pymesh\newline\newline
- Define nodes and elements\newline\newline
- In an image, each pixel is a square\newline\newline
- Can use image/pixels to create a list of nodes\newline\newline
- E.g. the corners of each pixel\newline\newline
- Elements are a bunch of nodes, can be done manually or with a python script\newline\newline
- Create a mesh that is compatible with sfepy\newline\newline
- See link on zoom meeting for layout of .vtk file\newline\newline
- Interested in a structured grid, look for format\newline\newline
- Best to use images that are given by Pratheek\newline\newline
- Can just do a simple diffusion problem, solving the diffusion equation\newline\newline
- Does not need to be too complex\newline\newline
- Noisy images cause problems in simulations\newline\newline
- The images provided are clean, does not require much preprocessing\newline\newline
- Do not need to put in the material parameters\newline\newline
- Getting the meshes to work is important\newline\newline
- Scalar diffusion equation could be considered or more complex ones e.g. thermal elasticity\newline\newline
- Can do a simulation in a region of interest of microstructure, not the entire image\newline\newline
- Finer mesh in the interesting areas, coarser mesh elsewhere\newline\newline
- Attend the 20th and 27th May workshops for attendance (impacts our mark)\newline\newline
- If any issues with meshing, contact Pratheek to go through a tutorial \newline\newline




\end{document}